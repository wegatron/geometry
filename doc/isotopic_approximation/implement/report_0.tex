%%%%%%%%%%%%%%%%%%%%%%%%%%%%%%%%%%%%%%%%%%%%%%%%%%%%%%%%%%%%%%%%%%%%%%%%%%%
%
% Template for a LaTex article in English.
%
%%%%%%%%%%%%%%%%%%%%%%%%%%%%%%%%%%%%%%%%%%%%%%%%%%%%%%%%%%%%%%%%%%%%%%%%%%%

\documentclass{article}

% AMS packages:
\usepackage{amsmath, amsthm, amsfonts}
%%\usepackage{algorithm}
\usepackage[]{algorithm2e}
\usepackage[hyperref, UTF8]{ctex}
\usepackage[noend]{algpseudocode}
\usepackage{graphicx}
\usepackage{subcaption}
\usepackage[top=0.8in, bottom=0.8in, left=1in, right=1in]{geometry}
\graphicspath{ {images/} }

% Theorems
%-----------------------------------------------------------------
\newtheorem{thm}{Theorem}[section]
\newtheorem{cor}[thm]{Corollary}
\newtheorem{lem}[thm]{Lemma}
\newtheorem{prop}[thm]{Proposition}
\theoremstyle{definition}
\newtheorem{defn}[thm]{Definition}
\theoremstyle{remark}
\newtheorem{rem}[thm]{Remark}

\makeatletter
\def\BState{\State\hskip-\ALG@thistlm}
\makeatother
%\newcommand*{\rom}[1]{\expandafter\@slowromancap\romannumeral #1@}
\newcommand{\rom}[1]{\uppercase\expandafter{\romannumeral #1\relax}}
% Shortcuts.
% One can define new commands to shorten frequently used
% constructions. As an example, this defines the R and Z used
% for the real and integer numbers.
%-----------------------------------------------------------------
\def\RR{\mathbb{R}}
\def\ZZ{\mathbb{Z}}

% Similarly, one can define commands that take arguments. In this
% example we define a command for the absolute value.
% -----------------------------------------------------------------
\newcommand{\abs}[1]{\left\vert#1\right\vert}

% Operators
% New operators must defined as such to have them typeset
% correctly. As an example we define the Jacobian:
% -----------------------------------------------------------------
\DeclareMathOperator{\Jac}{Jac}

%-----------------------------------------------------------------
\title{Isotopic Approximation within a Tolerance Volume}
\author{wegatron\\
  %% \small Dept. Templates and Editors\\
  %% \small E12345\\
  %% \small Spain
}

\begin{document}
\maketitle

%% \abstract{Compiling Embedded\_thin\_shell progress}
\section{Bad 3d delaunay triangulation}
  \begin{figure}[h]
    \begin{subfigure}[b]{0.4\textwidth}
      \includegraphics[width=\textwidth]{bad_surface0.png}
      \caption[现象]{bi+zero\_surface的四面体化}
    \end{subfigure}
    \begin{subfigure}[b]{0.4\textwidth}
      \includegraphics[width=\textwidth]{bad_surface1.png}
      \caption[原因]{zero\_surface和bi相交}
    \end{subfigure}
  \end{figure}

  \begin{figure}[h]
    \begin{subfigure}[b]{0.4\textwidth}
      \includegraphics[width=\textwidth]{bad_inner_space0.png}
      \caption[现象]{bo+zero\_surface的四面体化}
    \end{subfigure}
    \begin{subfigure}[b]{0.4\textwidth}
      \includegraphics[width=\textwidth]{bad_inner_space1.png}
      \caption[原因]{bad inner space:teaport不连续,有洞}
    \end{subfigure}
  \end{figure}

  \section{Old Implementation}
  %% 理想情况下是:根据原模型(zero\_surface)做生成内外壳,并在保存内外壳的三角化的情况下做一个delaunay triangulation.
  %% 我原来的做法是:
  \begin{itemize}
  \item (bo+zero\_surface) + (bi+zero\_surface) 3D delaunay triangulation.
  \item Do edge collapse in zero\_surface.
  \end{itemize}
  problem: small number of edge collapsable number. Very large space to simlify.
  \section{Implementation steps}
  \begin{itemize}
    \item Do 3D delaunay triangulation with $p_{bi}+p_{bo}$, as the result of papers Refinement.
    \item Simplicial Tolerance:do edge collapse with bi-bi edges or bo-bo edges.
    \item Zero-set edge collapse.
    \item All edge edge collapse and then do Zero-set edge collapse again, if possible.
  \end{itemize}
  \subsection{Simplicial Tolerance}
  \begin{figure}[h]
    \includegraphics[width=\textwidth]{simplicial_tolerance.png}
    \caption{Do edge collapse with bi-bi edges or bo-bo edges, using the sample point as candicate.}
  \end{figure}
  The merge point choice by three condition:
  \begin{itemize}
    \item lay in the kernel region.
    \item do not leading to errors in the classification of S.
    \item the point with maximum error.
  \end{itemize}

  \begin{figure}[h]
      \begin{subfigure}[b]{0.5\textwidth}
        \includegraphics[width=\textwidth]{keep_classification0}
        \caption[a]{keep classification of S}
      \end{subfigure}
      \begin{subfigure}[b]{0.5\textwidth}
        \includegraphics[width=\textwidth]{keep_classification1}
        \caption[b]{keep classification of S}
      \end{subfigure}
  \end{figure}

  \begin{algorithm}[H]
    \KwData{Tet mesh with sample point on bi and bo}
    \For{e $\in$ bo} {
      get all the sample points in its one-ring tets. all\_sp\_;\\
      get the points in kernel region. candi\_sp\_;\\
      \KwData{merge\_point, max\_error=-1}
      \For{each sample point $p \in candi\_sp\_$} {
        calculate all the sample point's error if edge collapse to p.\\
        \If{keep classification of S and error(p) $>$ max\_error} {
          merge\_point=p\\
          max\_error=error(p)
        }
      }
    }
   \caption{Simplicial tolerance}
\end{algorithm}
\end{document}
