%%%%%%%%%%%%%%%%%%%%%%%%%%%%%%%%%%%%%%%%%%%%%%%%%%%%%%%%%%%%%%%%%%%%%%%%%%%
%
% Template for a LaTex article in English.
%
%%%%%%%%%%%%%%%%%%%%%%%%%%%%%%%%%%%%%%%%%%%%%%%%%%%%%%%%%%%%%%%%%%%%%%%%%%%

\documentclass{article}

% AMS packages:
\usepackage{amsmath, amsthm, amsfonts}
\usepackage{algorithm}
\usepackage[noend]{algpseudocode}
\usepackage{graphicx}
\graphicspath{ {images/} }

% Theorems
%-----------------------------------------------------------------
\newtheorem{thm}{Theorem}[section]
\newtheorem{cor}[thm]{Corollary}
\newtheorem{lem}[thm]{Lemma}
\newtheorem{prop}[thm]{Proposition}
\theoremstyle{definition}
\newtheorem{defn}[thm]{Definition}
\theoremstyle{remark}
\newtheorem{rem}[thm]{Remark}

\makeatletter
\def\BState{\State\hskip-\ALG@thistlm}
\makeatother
%\newcommand*{\rom}[1]{\expandafter\@slowromancap\romannumeral #1@}
\newcommand{\rom}[1]{\uppercase\expandafter{\romannumeral #1\relax}}
% Shortcuts.
% One can define new commands to shorten frequently used
% constructions. As an example, this defines the R and Z used
% for the real and integer numbers.
%-----------------------------------------------------------------
\def\RR{\mathbb{R}}
\def\ZZ{\mathbb{Z}}

% Similarly, one can define commands that take arguments. In this
% example we define a command for the absolute value.
% -----------------------------------------------------------------
\newcommand{\abs}[1]{\left\vert#1\right\vert}

% Operators
% New operators must defined as such to have them typeset
% correctly. As an example we define the Jacobian:
% -----------------------------------------------------------------
\DeclareMathOperator{\Jac}{Jac}

%-----------------------------------------------------------------
\title{Bilateral Normal Filter}
\author{ShengweiZHANG\\
  %% \small Dept. Templates and Editors\\
  %% \small E12345\\
  %% \small Spain
}

\begin{document}
\maketitle

%% \abstract{Compiling Embedded\_thin\_shell progress}
\section{Abstract}
Bilateral filter for processing a normal field defined over an input mesh, smooth the face normal and update the vertexes according the face normal.
\section{The Filter}
\subsection{Normal Filter}
Just directly apply the traditional bilateral filter to the normal field:
\begin{equation}
  n_i' = K_i \sum_{j\in N(i)} \zeta_{ij} W_c(\parallel c_j-c_i\parallel) W_s(\parallel n_j - n_i \parallel) n_j
\end{equation}
where $c_i, n_i$  is the center and normal of face i, $K_i$ is the nomalize factor. $W_c, W_s$ are Guass function, $\zeta_{ij}$ is the weight to account for the influence from surface sampling rate, choose the area $S_j$ of the face j:
\begin{equation}
  \begin{aligned}
    K_i &= \sum_{j\in N(i)} \zeta_{ij} W_c(\parallel c_j-c_i\parallel) W_s(\parallel n_j - n_i \parallel)  [paper]\\
    K_i &= 1/norm(\sum_{j\in N(i)} \zeta_{ij} W_c(\parallel c_j-c_i\parallel) W_s(\parallel n_j - n_i \parallel) n_j) [mine] \\
    W_c(\parallel c_j-c_i \parallel) &= exp(-\frac{(\parallel c_j-c_i \parallel)^2}{2\sigma_c^2})\\
    W_s(\parallel n_j-n_i \parallel) &= exp(-\frac{(\parallel n_j-n_i \parallel)^2} {2\sigma_s^2})\\
    \zeta_{ij} = S_j
  \end{aligned}
\end{equation}
\subsection{Updating Vertexes}
\begin{equation}
  v_i' = v_i + \frac{1}{18} \sum_{v_j \in N_i^v} \sum_{ (v_i, v_j) \in \Delta F_m} n_m'(n_m' \cdot (v_j - v_i))
\end{equation}
\section{Parameter}
The $\sigma_s$'s influence on $W_s$ is:
\begin{equation}
  \begin{aligned}
      W_s(\parallel n_j-n_i \parallel) &= exp(-\frac{(\parallel n_j-n_i \parallel)^2} {2\sigma_s^2})\\
      &= exp(-\frac{n_j^2 + n_i^2  - 2n_jn_i} {2\sigma_s^2}) \\
      &=exp(-\frac{1-cos(\theta)} {\sigma_s^2})
   \end{aligned}
\end{equation}
\begin{figure}[H]
\includegraphics[width=12cm]{theta_inf}
 \caption[The $\sigma_s$'s influence on $W_s$]
   {Blue $\sigma_s$ is 0.3, Red one is 0.9, the horizontal axis is the angle of face normal $n_i$ and $n_j$.}
\centering
\end{figure}
From the Figure 1, as we want to preserve the feature of the model, I set $\sigma_s = 0.3$. For $\sigma_c$ set as the paper's saying: The $\sigma_c$  using the average distance of all adjacent facets in an input mesh generally works the best in our experiments. \\
For number of iterations for smoothing face normals usually 3, however if noise is in high level, this could be larger.\\
For number of iterations for updating vertex positions 15 is suitable in most cases.
\section{Details}
One-ring face $N_{\rom{1}}$ is a set of faces that share edges with $f_i$, and the second type $N_{\rom{2}}$ is a set of faces that common vertices with $f_i$. As the paper's author speculate $N_{\rom{2}}$ is able to better characterize sharp features which is more suitable for CAD-like models. Here I using $N_{\rom{1}}$.
\begin{algorithm}
\caption{Filtering Normal One Iteration}\label{euclid}
\begin{algorithmic}
 \Procedure{Init Procedure}{}
 \State caculate each face's area
 \State caculate each face's normal matrix: fnormal
 \State tmp\_fnormal = fnormal
 \EndProcedure
 \For{each face i in mesh}
 \State query 1-ring face set of face i
 \State new\_ni = (0,0,0)
 \For{face j in face set}
 \State caculate $W_c$  and $W_s$
 \State new\_ni += $S_j*W_c*W_s*fnormal(colon(), j)$
 \EndFor
 \State tmp\_fnormal(colon(), i) = new\_ni / norm(new\_ni)
\EndFor
\State fnormal = tmp\_fnormal
\end{algorithmic}
\end{algorithm}
\section{Limitations}
As the author says, the iterative method cannot guarantee convergence, more iterations lead to smoother normal fields (i.e., possibly larger normal errors). I find more iterations lead to sharp edges which is not expected(see the cube's case). So this method has big limitations on mesh with high level noise.
\section{Thinking}
May be if we want to filter high level noise, we can't simply classify noise and feature just by face normal's difference by $W_s$,
we need to extended to n-ring, and find a more intellectual way to tell them apart.
%% \begin{equation}\label{eq:area}
%%   S = \pi r^2
%% \end{equation}
%% One can refer to equations like this: see equation (\ref{eq:area}). One can also
%% refer to sections in the same way: see section \ref{sec:nothing}. Or
%% to the bibliography like this: \cite{Cd94}.

%% \subsection{Subsection}\label{sec:nothing}

%% More text.

%% \subsubsection{Subsubsection}\label{sec:nothing2}

%% More text.

%% % Bibliography
%% %-----------------------------------------------------------------
%% \begin{thebibliography}{99}

%% \bibitem{Cd94} Author, \emph{Title}, Journal/Editor, (year)

%% \end{thebibliography}

\end{document}
